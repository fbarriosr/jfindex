
\documentclass[preprint,landscape,12pt]{elsarticle}
\usepackage[margin=0.5in]{geometry}
\usepackage{color}
\usepackage[version=3]{mhchem} % Formula subscripts using \ce{}
\usepackage[T1]{fontenc}       % Use modern font encodings
\usepackage{subcaption}
\usepackage{latexsym}
\usepackage{amssymb,amsmath}
\usepackage{color}
\usepackage{lineno,hyperref}
\modulolinenumbers[5]
\usepackage{adjustbox}
\newcommand{\hilight}[1]{\colorbox{yellow}{#1}}
\usepackage{rotating}
\usepackage{multirow}
\usepackage{commath}
\usepackage{booktabs,caption}
\usepackage{mathptmx}      % use Times fonts if available on your TeX system
\usepackage{threeparttable}

\journal{JOURNAL}
\begin{document}
	\begin{table}
		\caption{ Tabla hartree}
		\centering
		\footnotesize
		\begin{tabular}{lrrrrrrrr}
			\hline
			\textbf{Density}    & $\varepsilon_{_{\mathrm{H}}}$	& $\varepsilon_{_{\mathrm{L}}}$  & $\varepsilon_{_{\mathrm{S}}}$& HL$Gap$ & $J(I)$ & $J(A)$ & $J(\mathrm{HL})$  & \textbf{$\left|\Delta\,\mathrm{SL}\right|$}  \\
			\textbf{Functional} &   &  &     &   &  &  &  &  \\
			\hline \hline 

RPBE0DH(B10C2) & -0.64478 & -0.45946 & -0.3997 & 0.18532 & 0.03094 & 0.0296 & 0.04282 & 0.05976\\

	 		\hline
		\end{tabular}
			\label{tab:hartree}
	\end{table}
\end{document}
\endinput

